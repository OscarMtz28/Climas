\documentclass[12pt]{article}
\usepackage[utf8]{inputenc}
\begin{document}
\begin{titlepage}
    \centering
    \vspace*{3cm}
    \Huge{\textbf{Práctica 1: Servicios del clima}}\\[0.5cm]
    \Large{Oscar Martinez Barrales}\\[0.5cm]
    \Large{\textbf{Universidad Autónoma Metropolitana}}\\[0.5cm]
    \Large{\textbf{Unidad Cuajimalpa}}\\[0.5cm]
    \Large{Sistemas Distribuidos} 
\end{titlepage}
\section*{Descripción del problema}

Una ciudad busca monitorear de manera constante algunos atributos del clima como lo son:
\begin{itemize}
    \item Temperatura
    \item Humedad
    \item Calidad del aire
\end{itemize}
Para ello buscan implementar un sistema a base de microservicios que les permita obetner esta informacion
de manera confiable. Tambien se encargara de enviar alertas de temperatura cuando 
se superen los limites establecidos.
\section*{Solucion proupuesta}
Para poder solucionar el problema propuesto se ocupo la implementacion de microservicios 
los cuales se encargan de diferentes tareas:
\begin{itemize}
    \item \textbf{sensor service}: Este microservicio se encarga de simular los sensores y enviar los datos en formato JSON
    \item \textbf{collector service}: Este microservicio es el encargado de distribuir los datos al resto de microservicios
    \item \textbf{Storage service}: En este microservicio ligamos los datos recibidos por collector service y la lista de barrios de la ciudad
    \item \textbf{alert service}: Este microservicio es el encargado de hacer la validacion de la temperatura. Si esta excede de los limites establecidos enviara una alerta al sistema
\end{itemize}
\end{document}